%%%%%%%%%%%%%%%%%%%%%%%%%%%%%%%%%%%%%%%%%
% Short Sectioned Assignment LaTeX Template Version 1.0 (5/5/12)
% This template has been downloaded from: http://www.LaTeXTemplates.com
% Original author:  Frits Wenneker (http://www.howtotex.com)
% License: CC BY-NC-SA 3.0 (http://creativecommons.org/licenses/by-nc-sa/3.0/)
%%%%%%%%%%%%%%%%%%%%%%%%%%%%%%%%%%%%%%%%%

%----------------------------------------------------------------------------------------
%   PACKAGES AND OTHER DOCUMENT CONFIGURATIONS
%----------------------------------------------------------------------------------------

\documentclass[10pt,a4paper,spanish]{article}

% ---- Entrada y salida de texto -----

\usepackage[spanish]{babel} 
\usepackage[T1]{fontenc} % Use 8-bit encoding that has 256 glyphs
\usepackage[utf8]{inputenc}
\usepackage{cite}
% \usepackage{spreadtab}
% \usepackage{fourier} % Use the Adobe Utopia font for the document - comment this line to return to the LaTeX default
\usepackage[usenames, dvipsnames]{color}
\usepackage[table]{xcolor}
\usepackage{colortbl}
\usepackage[bookmarks=true,colorlinks=true,linkcolor=red,citecolor=blue]{hyperref}
% \usepackage{cite}
% \usepackage[official]{eurosym}
\usepackage{tikz}
% \usepackage{pgfplots}
% \pgfplotsset{compat=1.5}

\usepackage{subfigure}

% \usepackage{pseudocode}

% ---- Otros paquetes ----
\usepackage{enumerate}
\usepackage{amsmath,amsfonts,amsthm,amssymb} % Math packages
\usepackage{graphics,graphicx} %para incluir imágenes y notas en las imágenes
% Para hacer tablas comlejas
%\usepackage{multirow}
%\usepackage{threeparttable}

\usepackage[a4paper, margin=1.3in]{geometry}


\usepackage{sectsty} % Allows customizing section commands
\allsectionsfont{\centering \normalfont\bfseries\scshape} % Make all sections centered, the default font and small caps
\usepackage{fancyhdr}
\pagestyle{fancy}
%con esto nos aseguramos de que las cabeceras de capítulo y de sección vayan en minúsculas

\renewcommand{\sectionmark}[1]{%
      \markright{\thesection\ #1}}
\fancyhf{} %borra cabecera y pie actuales
\fancyhead[LE,RO]{{\bfseries Práctica 1}}
\fancyhead[LO]{\bfseries Marta Gómez}
\fancyfoot[C]{\thepage{}}
\renewcommand{\headrulewidth}{0.5pt}
\renewcommand{\footrulewidth}{0pt}
\addtolength{\headheight}{0.5pt} %espacio para la raya
\fancypagestyle{plain}{%
      \fancyhead{} %elimina cabeceras en páginas "plain"
      \renewcommand{\headrulewidth}{0pt} %así como la raya
}

\numberwithin{equation}{section} % Number equations within sections (i.e. 1.1, 1.2, 2.1, 2.2 instead of 1, 2, 3, 4)
\numberwithin{figure}{section} % Number figures within sections (i.e. 1.1, 1.2, 2.1, 2.2 instead of 1, 2, 3, 4)
\numberwithin{table}{section} % Number tables within sections (i.e. 1.1, 1.2, 2.1, 2.2 instead of 1, 2, 3, 4)

\setlength\parindent{0pt} % Removes all indentation from paragraphs - comment this line for an assignment with lots of text
\setlength{\parskip}{1ex plus 0.5ex minus 0.2ex}

\newcommand{\horrule}[1]{\rule{\linewidth}{#1}} % Create horizontal rule command with 1 argument of height

%----------------------------------------------------------------------------------------
%   TÍTULO Y DATOS DEL ALUMNO
%----------------------------------------------------------------------------------------

\title{
\normalfont \normalsize 
{\bf Redes y Sistemas Complejos} \\ Curso 2016-2017 \\ [25pt] % Your university, school and/or department name(s)
\horrule{0.5pt} \\[0.4cm] % Thin top horizontal rule
\huge \textsc{Práctica 1: \\ Análisis Preliminar y Visualización \\ Básica de una red de Facebook \\ con \textit{Gephi}} \\ % The assignment title
\horrule{2pt} \\[0.5cm] % Thick bottom horizontal rule
}

\author{\textit{Marta Gómez Macías}} %\\ \texttt{mgmacias95@correo.ugr.es} \\ 75929776Z \\[0.5cm]

% \date{\normalsize\today} % Incluye la fecha actual

% \usepackage{pdflscape}

%----------------------------------------------------------------------------------------
% DOCUMENTO
%----------------------------------------------------------------------------------------

\begin{document}
%Cambiar Cuadros por Tablas y lista de...
\renewcommand{\listtablename}{Índice de tablas}
\renewcommand{\tablename}{Tabla} 

\begin{titlepage}
\begin{center}
\includegraphics[width=0.2\textwidth]{../../ugr}

\normalfont \normalsize 
{\bf Redes y Sistemas Complejos} \\ Curso 2015-2016 \\ [25pt] % Your university, school and/or department name(s)
\horrule{0.5pt} \\[0.4cm] % Thin top horizontal rule
{\huge \textsc{Práctica 1: \\ Análisis Preliminar y Visualización \\ Básica de una red de Facebook \\ con \textit{Gephi} \\}} % The assignment title
\horrule{2pt} \\[0.5cm] % Thick bottom horizontal rule

{\Large \textit{Marta Gómez Macías} \\ \texttt{mgmacias95@correo.ugr.es} \\ 75929776Z \\[0.5cm]

\date{\today}} % Incluye la fecha actual
\end{center}
\end{titlepage}

\tableofcontents % para generar el índice de contenidos

% \listoffigures

% \listoftables

\section{Descripción de la red}

La red a analizar ha sido extraída del grupo de \textit{Facebook} llamado \href{https://www.facebook.com/groups/bastadeabusosconnuestrasfotos/}{\textbf{STOP CLAUSULAS ABUSIVAS A LOS FOTÓGRAFOS}}.

% \subsection{Grafo de la red}

En la \hyperref[completo]{Figura \ref*{completo}} vemos una representación del grafo completo de la red y en la \hyperref[conexa]{Figura \ref*{conexa}}, una representación de la componente gigante. Ambas son prácticamente iguales, salvo por dos nodos que están conectados entre sí más alejados del resto. 

Para generar la visualización he optado finalmente por el algoritmo de visualización \textbf{Yifan Hu}, con los parámetros que trae por defecto. El \textbf{tamaño} de los nodos ha sido calculado por su \textbf{grado}. A la hora de \textbf{colorear} los nodos, he usado un atributo de los datos llamado \textbf{made\_posts} que mide el número de posts que ha hecho ese usuario. 

% \begin{figure}[!h]
%     \mbox{
%     \subfigure[Representación del grafo completo de la red]{
%     \includegraphics[width=0.5\textwidth]{1}
%     \label{completo}
%     }
%     \subfigure[Representación de la componente gigante del grafo de la red] {
%     \includegraphics[width=0.5\textwidth]{2}
%     \label{conexa}
%     }
%     }
%     \caption{Visualización de la red}
%     \label{red}
% \end{figure}

\begin{figure}[!h]
    \centering
    \includegraphics[width=\textwidth]{1}
    \caption{Representación del grafo completo de la red}
    \label{completo}
\end{figure}

\begin{figure}[!h]
    \centering
    \includegraphics[width=\textwidth]{2}
    \caption{Representación de la componente gigante del grafo de la red}
    \label{conexa}
\end{figure}

\begin{figure}[!h]
    \centering
    \includegraphics[width=0.5\textwidth]{3}
    \caption{Proporción de nodos según el número de posts que han hecho.}
    \label{proporcion}
\end{figure}


% \subsection{Datos estadísticos sobre la red}\label{sec:estadistica}

\begin{table}[!h]
\centering
\begin{tabular}{| c | c |}
\hline
Número de nodos ($N$) & $815$ \\
\hline
Número de enlaces ($L$) & $1075$ \\
\hline
Número máximo de enlaces ($L_{max}$) & $331705$ \\
\hline
Densidad del grafo ($\frac{L}{L_{max}}$) & $0.003$ \\
\hline
Grado medio ($<k>$) & $2.683$ \\
\hline
Diámetro ($d_{max}$) & $6$ \\
\hline
Distancia media ($d$) & $3.543$ \\
\hline
Coeficiente medio de clustering ($<C>$) & $0.105$ \\
\hline
Número de componentes conexas & $2$ \\
\hline
Número de nodos de la componente gigante & $813$ ($99.75\%$)\\
\hline
Número de aristas de la componente gigante & $1074$ ($99.91\%$)\\
\hline
\end{tabular}
\caption{Datos estadísticos sobre la red}
\label{tabla}
\end{table}

% \subsection{Gráficos de distribuciones de la red}

\begin{figure}[!h]
    \centering
    \includegraphics[width=0.75\textwidth]{../coeficiente_medio_clustering/clustering-coefficient}
    \caption{Gráfico de distribución del coeficiente medio de clústering, $<C>$}
    \label{coeficientemedioclustering}
\end{figure}

\begin{figure}[!h]
    \centering
    \includegraphics[width=0.75\textwidth]{../diametro_red/Betweenness-Centrality-Distribution}
    \caption{Gráfico de distribución del coeficiente de intermediación.}
    \label{intermediacion}
\end{figure}

\newpage
\section{Análisis de la red}

Observando la \hyperref[completo]{Figura \ref*{completo}}, se ve que hay dos usuarios que participan muchísimo en todos los posts del grupo. Después vemos nodos con tamaño medio, que supongo que corresponden con los usuarios que han publicado un post y muchos nodos conectados a estos, que corresponden con los usuarios que iteraccionan con ese post. Observando los datos de la \hyperref[tabla]{Tabla \ref*{tabla}}, vemos que el grado medio es 2 ($<k>=2.683$), lo que significa que, en media, \textbf{cada usuario interactúa con otros 2 usuarios}. Observando el diámetro ($d_{max} = 6$) y la distancia media ($d = 3.546$) vemos que todos los usuarios están relacionados entre sí en, como mucho, 6 interacciones pero en media, este número desciende a 3. Con sólo estos datos ya podemos ver que es un \textbf{grupo interconectado}, de hecho, la componente gigante del grupo representa el $99.75\%$ en cuando a nodos y el $99.91\%$ en cuanto a aristas. Podemos comprobar esta afirmación observando el grafo (\hyperref[conexa]{Figura \ref*{conexa}}). 

Si observamos los datos estadísticos (\hyperref[tabla]{Tabla \ref*{tabla}}), ya sabemos que el grafo \textbf{no es completo}, ya que $L_{max} > L$. Además, teniendo en cuenta que la densidad del grafo es $d = 0.003$ y que el coeficiente medio de clústering es $<C> = 0.105$, vemos que además de no ser completo \textbf{hay poca interconexión entre los nodos vecinos}: hay nodos que reciben muchas interacciones y otros que sólo interactúan con un único nodo, es decir, hay pocos usuarios que publiquen posts pero muchos que comentan e interactúan con los distintos posts. Podemos comprobar estas dos afirmaciones observando el grafo (\hyperref[completo]{Figura \ref*{completo}}) y la distribución de colores por posts escritos (\hyperref[proporcion]{Figura \ref*{proporcion}}). 

En la \hyperref[intermediacion]{Figura \ref*{intermediacion}} vemos la distribución del coeficiente de intermediación. Como se ve en la \hyperref[conexa]{Figura \ref*{conexa}}, la mayoría de usuarios del grupo están conectados entre sí por sus interacciones con los usuarios que han escrito algún post. Por tanto, son estos pocos usuarios los que tienen mayor intermediación y aparecen en la parte baja del gráfico y el resto, están representados por un único punto en la coordenada $(0,650)$.

En resumen, el grafo está interconectado entre sí por los usuarios más populares del grupo, pero en realidad, la tónica general de la mayoría de usuarios es interactuar con un usuario y ya está.

% Con el coloreado del grafo (\hyperref[completo]{Figura \ref*{completo}}) según el número de posts escritos por cada usuario confirmo lo que digo en el párrafo anterior: los nodos con mayor grado corresponden con los usuarios que han hecho algún post y los nodos con un grado pequeño, con los usuarios que interaccionan con dicho post. En la \hyperref[proporcion]{Figura \ref*{proporcion}}, vemos la proporción de usuarios según el número de posts que han escrito. El usuario que más posts ha escrito se ha coloreado en azul (se sitúa en el centro del grafo) y los dos usuarios con mayor grado, han escrito dos posts pero han iteractuado más con el resto de usuarios. Los nodos coloreados en morado son los que no han escrito ningún post, y representan la mayoría.


Además, los gráficos de distribución también apoyan esto, ya que en , como vemos en la \hyperref[coeficientemedioclustering]{Figura \ref*{coeficientemedioclustering}},la mayoría de nodos (750) tienen $<C_i> = 0$ y sólo unos pocos tienen un coeficiente algo mayor. Los usuarios con $<C_i> = 0$ son todos los que quedan en la periferia de la componente gigante conectados a un usuario que ha escrito un post y los nodos que tienen $<C_i> > 0$, la minoría, se corresponden con los usuarios activos en el grupo de los que hablábamos antes. 

Para terminar de confirmar esto, vemos que observando los datos estadísticos (\hyperref[tabla]{Sección \ref*{tabla}}), la red tiene una densidad muy baja ($d=0.003$) y un coeficiente de clústering muy bajo ($<C> = 0.105$). 

Por tanto, a modo de resumen, podemos decir que en este grupo hay tres tipos de usuarios:

\begin{enumerate}[$\bullet$]
    \item Usuarios que escriben posts y pueden también interactuar con los posts de los demás.
    \item Usuarios que no escriben posts pero interaccionan bastante a menudo con los posts de los demás.
    \item Usuarios que interaccionan con sólo un post.
\end{enumerate}

\end{document}